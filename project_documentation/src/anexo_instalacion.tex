\section{Instalación y despliegue de la aplicación}

En esta sección se va a hacer una guía sobre los pasos necesarios para llevar a cabo una correcta instalación de servicios y dependencias para la aplicación \texttt{SIVIRA}.

\subsection{Instalación mínima}

Se instalarán los componentes mínimos y necesarios para que la aplicación pueda ejecutarse correctamente. Estos son:

\begin{itemize}
\item RabbitMQ.
\item SQLite3.
\item Dependencias de Python3.
\item Dependencias para el servidor de streaming.
\end{itemize}


\textbf{Servicio RabbitMQ}

RabbitMQ es un broker de mensajes de código abierto. Es necesario para poder gestionar las colas de las tareas asíncronas de la aplicación utilizadas por \texttt{celery}.

Se puede instalar fácilmente usando el siguiente comando:

\vspace{-1.4cm}

\begin{verbatim}

# sudo apt-get install rabbitmq-server

\end{verbatim}

\vspace{-1.4cm}

Por defecto, \texttt{RabbitMQ} escuchará en el puerto 5672 en todas las interfaces disponibles. Puede comprobar si el servicio está disponible con:


\vspace{-1.4cm}

\begin{verbatim}

# sudo service rabbitmq-server status

\end{verbatim}

\vspace{-1.4cm}

\textbf{SQLite3}

Para almacenar los estados de las tareas asíncronas, es necesario especificar un backend de almacenamiento. En este caso, se ha seleccionado \texttt{SQLite} ya que es muy ligero y fácil de usar y configurar.

Puede instalar fácilmente este servicio, usando el comando:

\vspace{-1.4cm}

\begin{verbatim}

# sudo apt-get install sqlite3

\end{verbatim}

\vspace{-1.4cm}

Una vez instalado, compruebe que puede acceder al terminal sqlite con el comando:

\vspace{-1.4cm}

\begin{verbatim}

# sqlite3

\end{verbatim}

\vspace{-1.4cm}

\textbf{Dependencias de Python}

Para instalar las dependencias de Python, se recomienda crear un entorno virtual. Puede crear uno rápidamente usando el comando:

\vspace{-1.4cm}

\begin{verbatim}

# python3 -m venv ./venv

\end{verbatim}

\vspace{-1.4cm}

A continuación, puede activar ese entorno utilizando:

\vspace{-1.4cm}

\begin{verbatim}

# source venv/bin/activate

\end{verbatim}

\vspace{-1.4cm}

Después, puede instalar todas las dependencias necesarias utilizando el siguiente comando:

\vspace{-1.4cm}

\begin{verbatim}

# pip3 install -r minium_requirements.txt

\end{verbatim}

\vspace{-1.4cm}

\textbf{Dependencias para el servidor de streaming}

Para la transmisión del vídeo en tiempo real, es necesario instalar el siguiente paquete de códecs para websockets.

\vspace{-1.4cm}

\begin{verbatim}

# sudo apt-get install ffmpeg python3-ws4py

\end{verbatim}

\vspace{-1.4cm}


