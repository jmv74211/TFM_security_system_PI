

\fancypagestyle{miEstilo6}{
   \lhead{6. Presupuesto}
   \rhead{Página \thepage}
   \lfoot{}
   \cfoot{}
   \rfoot{}
}

\pagestyle{miEstilo6}

\section{Presupuesto}

En esta sección se va a realizar un presupuesto estimado y justificado acerca del coste del proyecto. Adicionalmente, se realizará también una estimación de costes y beneficios que conllevaría la puesta en marcha de este proyecto al mercado laboral.

\subsection{Presupuesto desglosado por conceptos}

\begin{table}[H]
\centering
\begin{tabular}{|l|l|}
\hline
\rowcolor[HTML]{4F81BD} 
{\color[HTML]{FFFFFF} \textbf{Gastos elegibles}} & {\color[HTML]{FFFFFF} \textbf{Importe solicitado}} \\ \hline
\rowcolor[HTML]{EFEFEF} 
{\color[HTML]{FE0000} Gasto de personal} & {\color[HTML]{FE0000} 9202\euro} \\ \hline
\hspace{1cm}Total gastos de contratación & {\color[HTML]{F56B00} 9202\euro} \\ \hline
\rowcolor[HTML]{EFEFEF} 
{\color[HTML]{FE0000} Gastos de ejecución} & {\color[HTML]{FE0000} 733.5\euro} \\ \hline
Costes de adquisición de material inventariable & {\color[HTML]{F56B00} 298.5\euro} \\ \hline
\hspace{1cm}1 portátil durante 3 meses & 145\euro \\ \hline
\hspace{1cm}1 Kit Raspberry PI & 55\euro \\ \hline
\hspace{1cm}1 Cámara & 6\euro \\ \hline
\hspace{1cm}1 Sensor de movimiento & 1\euro \\ \hline
\hspace{1cm}1 Caja para la Raspberry PI & 11\euro \\ \hline
\hspace{1cm}1 Cableado & 2.5\euro \\ \hline
\hspace{1cm}1 Licencia 3 meses Visual Paradigm & 18\euro \\ \hline
\hspace{1cm}1 Licencia 3 meses Pycham & 60\euro \\ \hline
Costes de suministros... & {\color[HTML]{F56B00} 435\euro} \\ \hline
\hspace{1cm}Alquiler estancia & 150\euro \\ \hline
\hspace{1cm}Gasto de luz & 90\euro \\ \hline
\hspace{1cm}Gasto de agua & 80\euro \\ \hline
\hspace{1cm}Gasto de internet & 95\euro \\ \hline
\hspace{1cm}Gastos de consumibles varios: Café... & 20\euro \\ \hline
\rowcolor[HTML]{EFEFEF} 
{\color[HTML]{FE0000} Gastos complementarios} & {\color[HTML]{FE0000} 30\euro} \\ \hline
\hspace{1cm}Transporte para reuniones de revisión & 30\euro \\ \hline
\rowcolor[HTML]{EFEFEF} 
\multicolumn{1}{|r|}{\cellcolor[HTML]{EFEFEF}\textbf{Coste total}} & {\color[HTML]{FE0000} 9965.5\euro} \\ \hline
\end{tabular}
\end{table}

\newpage

\subsection{Justificación del presupuesto}

\textbf{Gastos de personal}

El precio hora bruto que se paga a un ingeniero informático se ha establecido en 17\euro

Dado que este proyecto se ha realizado de forma individual, los gastos de contratación se corresponderían solo con la suma del salario bruto más las deducciones a pagar. Dicho salario bruto vendría dado por la siguiente fórmula:

\vspace{-0.7cm}

\[
NºHorasTotales * PrecioHora + SueldoProporcionalVacaciones
\]

El sueldo proporcional a las vacaciones se correspondería con dos días y medio de vacaciones por cada mes trabajado, por lo tanto se calcularía de la siguiente forma:

\vspace{-1cm}

\[
SueldoProporcionalVacaciones =  (3 meses * (2.5 \text{días/mes} *  \text{8h/día})) *17 \text{\euro}/hora = 1020 \text{\euro}
\]

\vspace{-0.3cm}

Si sustituimos los valores dados en la tabla obtenemos lo siguiente:

\vspace{-0.7cm}

\[
\text{\textit{Salario bruto =} }341h*17\text{\euro/h}+1020\text{\euro} = 6817\text{\euro}
\]

Finalmente, el gasto por personal total del proyecto sería:

\vspace{-0.7cm}

\[
\text{\textit{Gasto de personal}} = \text{\textit{salario bruto}} + 35\text{\%} \text{\textit{deducción}} = 6817\text{\euro} + 35\text{\%} = 9202\text{\euro}
\]

\textbf{Gastos de ejecución}

Los costes de ejecución vienen dados por la siguiente fórmula:

$\text{\textit{Coste}} = \text{\textit{cantidad}} * \text{\textit{precio}} * \dfrac{\text{\textit{Tiempo uso}}}{\text{\textit{Periodo amortización}}}$

A continuación se desglosan el conjunto de gastos

\begin{itemize}
\item Portátil = $ 1u* 1750\text{\euro} * (3 meses/36 meses) = 145\text{\euro} $
\item Kit Raspberry PI = $ 1u*55\text{\euro} * 1 = 55\text{\euro} $
\item Cámara = $ 1u* 6\text{\euro} * 1 = 6\text{\euro} $
\item Sensor de movimiento = $ 1u* 1\text{\euro} * 1 = 1\text{\euro} $
\item Caja para la Raspberry PI = $ 1u* 11\text{\euro} * 1 = 11\text{\euro} $
\item Cableado = $ 1u* 2.5\text{\euro} * 1 = 2.5\text{\euro} $
\item Licencia 3 meses Visual Paradigm = $ 1u* 18\text{\euro} = 18\text{\euro} $
\item Licencia 3 meses Pycharm. = $ 1u* 20\text{\euro} = 20\text{\euro} $
\end{itemize}

\begin{tabular}{|p{15.5cm}|}
	
	\hline
	
	\textit{ \textbf{*Nota:} Cuando el resultado del tiempo de uso entre el periodo de amortización vale 1, significa que el uso de dicho componente es únicamente dedicado para este proyecto.}
	\\
	\hline
	
\end{tabular}

\vspace{0.3cm}

Respecto a los gastos de suministros, los precios se han estimado en función de la media total que se pagan por dichos suministros fuera del periodo de desarrollo.

\textbf{Gastos Complementarios}

Durante el desarrollo de este proyecto, se han previsto un total de 3 visitas al cliente (en este caso el tutor del TFM). Se estima alrededor de 1 hora por visita más el gasto de transporte que hacen una estimación total de 30\euro.

\subsection{Estimación de costes y beneficios en el mercado laboral}

En la sección anterior, se ha podido comprobar el coste que ha supuesto el desarrollo de este proyecto, pero ¿qué pasaría si este proyecto se lanzara al mercado laboral?. A continuación se va a realizar una estimación de los costes y beneficios que esto supondría.

Antes de empezar a estimar posibles costes, se ha realizado un estudio acerca de las cuotas mensuales que suelen cobrar las principales compañías de alarmas y videovigilancia, y se ha podido ver que la cuota mensual suele rondar alrededor de los 30-40\euro (Por ejemplo aquí puede verse los \href{https://www.comparaiso.es/alarmas/empresas-seguridad/securitas-direct/precio}{precios de securitas-direct}).

También, mencionar que en un principio, la puesta en marcha y mantenimiento del servicio no supondría ningún coste externo de contratación de servicios (como cloud, licencias \ldots), sino que solo bastaría realizar una inversión inicial en el hardware (Raspberry PI y sus componentes). En base a este dato, se han realizado las siguientes estimaciones.

\textbf{Instalación del servicio}

El coste que supondría realizar la instalación del servicio sería el siguiente:

\begin{itemize}
\item \textbf{Coste real de los componentes}: 71\euro

\item \textbf{Gastos de instalación y desplazamiento}: 15\euro

\end{itemize}

\vspace{-0.4cm}

El presupuesto que se pediría al cliente para realizar la instalación sería de unos 100\euro. En este precio va incluido todo el hardware necesario, que pasaría a ser propiedad del cliente desde el primer momento.

El beneficio en esta etapa sería de \textbf{16\euro} (100 de presupuesto - 86 de coste). Esta cantidad es casi insignificante, y no aporta casi ninguna ganancia, pero el objetivo no es ganar una gran cantidad al realizar la instalación, sino por el mantenimiento del servicio. Por ello se ha intentado reducir el coste inicial con motivo de poder atraer el máximo número de clientes.

\textbf{Mantenimiento del servicio}

El mantenimiento del servicio se basaría en un soporte técnico y de ayuda a los clientes. Los clientes pagarían una cuota mensual a cambio de poder reportar incidencias para que puedan ser ayudados.

Este servicio no supondrá ningún coste externo más allá de la mano de obra, a no ser que sea por motivos de reparación del hardware (en cuyo caso el coste de reparación en componentes estaría a cargo del cliente).

Comparando con la competencia, se ha propuesto una \textbf{cuota mensual de \text{20\euro}}. Este precio intenta ser lo más competitivo posible para atraer el máximo número de clientes.

\newpage

\textbf{Marketing}

Otro de los principales aspectos a destacar es el marketing, es decir, cómo vender y promocionar este producto para hacerlo atractivo para los clientes. 

Esta tarea la delegaría en expertos de una agencia de marketing para que me pudieran guiar y aconsejar. Este coste supondría un total de unos \text{250\euro} mensuales.

\textbf{Conclusiones}

Este proyecto no tiene como objetivo principal el obtener una remuneración económica, sino que ha surgido con la idea de ser un proyecto de software libre, que todo el mundo pueda descargar, utilizar y mantener por sí solo.

Es cierto, que se podría intentar lanzar al mercado laboral, y por ello se ha realizado una posible estimación en esta sección.

Como se ha podido comprobar, para intentar ser lo más competitivo posible, se han estimado unos costes realmente bajos, por lo que para que fuera rentable, debería de tener un número considerable de clientes. Por ejemplo, a partir de los 100 clientes, ya se podría empezar a ganar cerca de unos \text{1000\euro} mensuales. También es cierto, que este proyecto se podría llevar simultáneamente con otros (hasta que se tenga una gran cantidad de clientes), ya que el mantenimiento puede no suponer mucho esfuerzo o por el contrario, dedicarse íntegramente e ir desarrollando nuevas funcionalidades.

También hay que tener en cuenta el coste inicial y de desarrollo del proyecto. Este ha sido alrededor de unos \text{10000\euro}, y que debería de irse recuperando a lo largo de los años. En resumen, una posible previsión podría ser la siguiente:

Ingresos mensuales = 2000\text{\euro} (100 clientes * 20\text{\euro} de cuota)

Gastos = 250\text{\euro} mensuales en marketing

Impuestos = 283\text{\euro} (Autónomo) + 420\text{\euro} (I.V.A) + 157.05\text{\euro} (IRPF) = 860.05\text{\euro}

Ganancia total = 2000\text{\euro} - (250\text{\euro} + 860.05\text{\euro}) = \textbf{889.95 \text{\euro}}

\newpage